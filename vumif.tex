\documentclass[12pt, a4paper, lithuanian]{article}

\usepackage[utf8x]{inputenc}
\def\LTfontencoding{L7x}
\usepackage[\LTfontencoding]{fontenc}
\usepackage[lithuanian]{babel}

\usepackage{VUMIF}

% \usepackage[mathcsdepttitle]{VUMIF} % --- matematinės informatikos katedros
%     titulinio puslapio formatavimas

% Titulinio puslapio reikalai
\vumifdept{Informatikos katedra}
\vumifpaper{Magistro baigiamasis darbas}
\title{Funkcinio-reaktyvaus programavimo taikymas įvykių kaupimo sistemose}
\engtitle{Functional Reactive Programming in Event Sourcing Systems}
\author{
    1 kurso 7 grupės studentas \\
    Žilvinas Kučinskas
}

\supervisor{Viačeslav Pozdniakov}
%\reviewer{XXX}
\date{Vilnius \\ 2013}

\begin{document}

\maketitle

\tableofcontents

\section{Magistro darbo objekto apžvalga bei tyrimo problemos aprašymas}

\subsection{Tyrimo objektas}

    Tyrimo objektas yra funkcinio-reaktyvaus programavimo bei įvykių kaupimo principai.

\subsection{Darbo tikslai ir uždaviniai}

    Darbo tikslas yra pritaikyti funkcinį-reaktyvų programavimą įvykių kaupimo sistemose.

    Siekiant šio tikslo, turi būti išspręsti šie uždaviniai:

\begin{itemize}
        \item išnagrinėti įvykių kaupimo principu paremtų programų sistemų kūrimą ir šio
                principo problemas;
        \item išnagrinėti funkcinio-reaktyvaus programavimo principą ir
            problemas, kurias šis principas sprendžia;
        \item įrodyti, kad funkcinį-reaktyvų programavimą įmanoma taikyti įvykių
            kaupimo sistemose;
        \item sukurti konkretizuotą kalbą (angl. domain specific language), apjungiančią funkcinio-reaktyvaus programavimo
            bei įvykių kaupimo principus;
        \item aprašyti konkretizuotuos kalbos kūrimo metodiką, apibrėžti gautų rezultatų apribojimus, suformuluoti iškilusias problemas bei paaiškinti jų priežastis.
\end{itemize}

\subsection{Tyrimo aktualumas}

    Funkcinis reaktyvus programavimas integruoja laiko tėkmę bei sudėtinius įvykius į funkcinį programavimą. Šis principas suteikia elegantišką būdą išreikšti skaičiavimus interaktyvių animacijų, robotikos, kompiuterinio vaizdavimo, vartotojo sąsajos ir modeliavimo srityse \cite[p. 4]{ELM:FRP}. Pagrindinės funkcinio reaktyvaus programavimo sąvokos:

\begin{itemize}
        \item signalai arba elgsena - reikšmės, besikeičiančios bėgant laikui;
        \item įvykiai - momentinių reikšmių kolekcijos arba laiko-reikšmės poros.
\end{itemize}

    Funkcinis-reaktyvus programavimas įgalina apsirašyti elgseną deklaratyviai naudojant imperatyvios programavimo kalbos struktūras \cite[p.1]{ElliottHudak97:Fran}. Elgsena ir įvykiai gali būti komponuojami kartu, išreikšti vienas per kitą. Funkcinis reaktyvus programavimas apibrėžia kaip signalai arba elgsena reaguoja į įvykius. \cite[p. 1]{Survey}

    Įvykių kaupimo principo esmė – objektas yra atvaizduojamas kaip įvykių seka. Kaip pavyzdį tai galima parodyti remiantis banko sąskaita. Tarkime vartotojas, banko klientas, turi 100 litų sąskaitos balansą. Tarkime vartotojas nusipirko prekę už 20 litų, tada įnešė į savo sąskaitą 15 litų ir galiausiai nusipirko tam tikrą paslaugą už 30 litų. Akivaizdu, jog turint šių įvykių seką, galima atvaizduoti dabartinę objekto būseną - tai yra 65 litai vartotojo sąskaitoje. Įvykių kaupimo principas užtikrina, jog visi būsenos pasikeitimai yra saugomi įvykių žurnale kaip įvykų seka \cite{DDD:Vernon}. Tai įgalina ne tik daryti užklausas šiems įvykiams, bet ir naudoti įvykių žurnalą atkurti būseną bet kuriuo laiko momentu praeityje.

    Pritaikius funkcinį-reaktyvų programavimą įvykių kaupimo principu paremtose sistemose būtų galima modeliuoti ne tik momentinius įvykius, tačiau turėti ir jų istoriją. Yra poreikis sukurti konkretizuotą kalbą (angl. domain specific language), kuri įgalintų paslėpti įvykių žurnalą (arba duomenų saugyklą). Pastarosios naudotojas galėtų orientuotis į pačią sprendžiamos srities problemą, nekreipdamas dėmesio į žemesnio lygio realizacijos detales. Šiuo atveju viename faile būtų galima deklaratyviai (ką kažkuri programos dalis turi daryti) apsirašyti elgseną, nutikus įvykiui, kartu su imperatyviomis(instrukcijos, kurios aprašo, kaip programos dalys atlieka savo užduotis) struktūromis.

\subsection{Tyrimo metodika}

    Darbo analitinėje dalyje bus naudojami tradiciniai bibliotekinio tyrimo metodai. Darbo tikslui pasiekti tiriamojoje dalyje bus pasirinkta konkreti funkcinė programavimo kalba (pvz.: Haskell, Scala) bei aprašoma kūrimo metodika.

    Papildomai I semestro metu ketinama užbaigti Erik Meijer, Martin Odersky, Roland Kuhn dėstomą reaktyvaus programavimo kursą (Coursera). Esant galimybėms žadama apsilankyti į funkcinį programavimą orientuotose konferencijose Scala Days (http://scaladays.org/), Scalar (http://scalar-conf.com/) ar kitose.

\subsection{Laukiami rezultatai}

    Magistrinio darbo metu planuojama išnagrinėti funkcinio-reaktyvaus programavimo ir įvykio kaupimo principus, įrodyti, jog šie principai gali būti panaudoti kartu bei suderinti, sukurti konkretizuotą kalbą (angl. domain specific language), apjungiančią šiuos principus, bei aprašyti kūrimo eigos metodiką, apibrėžti gautus rezultatus, suformuluoti apribojimus, iškilusias problemas bei paaiškinti jų priežastis.

\bibliography{references}

\end{document}
