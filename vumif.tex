\documentclass[12pt, a4paper, lithuanian]{article}

\usepackage[utf8x]{inputenc}
\def\LTfontencoding{L7x}
\usepackage[\LTfontencoding]{fontenc}
\usepackage[lithuanian]{babel}

\usepackage{VUMIF}

% \usepackage[mathcsdepttitle]{VUMIF} % --- matematinės informatikos katedros
%     titulinio puslapio formatavimas

% Titulinio puslapio reikalai
\vumifdept{Informatikos katedra}
\vumifpaper{Magistro baigiamasis darbas}
\title{Funkcinio-reaktyvaus programavimo taikymas įvykių kaupimo sistemose}
\engtitle{Functional Reactive Programming in Event Sourcing Systems}
\author{
    1 kurso 7 grupės studentas \\
    Žilvinas Kučinskas
}

\supervisor{Viačeslav Pozdniakov}
%\reviewer{XXX}
\date{Vilnius \\ 2013}

\begin{document}

\maketitle

\tableofcontents

\section{Magistro darbo objekto apžvalga bei tyrimo problemos aprašymas}

\subsection{Tyrimo objektas}

    Tyrimo objektas yra funkcinio-reaktyvaus programavimo bei įvykių kaupimo principai.

\subsection{Darbo tikslai ir uždaviniai}

    Darbo tikslas yra pritaikyti funkcinį-reaktyvų programavimą įvykių kaupimo sistemose.

    Siekiant šio tikslo, turi būti išspręsti šie uždaviniai:

\begin{itemize}
        \item išnagrinėti įvykių kaupimo principu paremtų programų sistemų kūrimą ir šio
                principo problemas;
        \item išnagrinėti funkcinio-reaktyvaus programavimo principą ir
            problemas, kurias šis principas sprendžia;
        \item įrodyti, kad funkcinį-reaktyvų programavimą įmanoma taikyti įvykių
            kaupimo sistemose;
        \item sukurti konkretizuotą kalbą (angl. domain specific language), apjungiančią funkcinio-reaktyvaus programavimo
            bei įvykių kaupimo principus;
        \item aprašyti konkretizuotuos kalbos kūrimo metodiką, apibrėžti gautų rezultatų apribojimus, suformuluoti iškilusias problemas bei paaiškinti jų priežastis.
\end{itemize}

\subsection{Tyrimo aktualumas}

Labai aktualu

\subsection{Tyrimo metodika}

    Darbo analitinėje dalyje bus naudojami tradiciniai bibliotekinio tyrimo metodai. Darbo tikslui pasiekti tiriamojoje dalyje bus pasirinkta konkreti funkcinė programavimo kalba (pvz.: Haskell, Scala) bei aprašoma kūrimo metodika.

    Papildomai I semestro metu ketinama užbaigti Erik Meijer, Martin Odersky, Roland Kuhn dėstomą reaktyvaus programavimo kursą (Coursera). Esant galimybėms žadama apsilankyti į funkcinį programavimą orientuotose konferencijose Scala Days (http://scaladays.org/) bei Scalar (http://scalar-conf.com/).

\subsection{Laukiami rezultatai}

Tampu super sajanu

\nocite{*}
\bibliography{references}

\end{document}
