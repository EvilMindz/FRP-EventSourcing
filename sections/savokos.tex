\begin{itemize}

	\item \textbf{Agregatas} (angl. aggregate) - DDD modelis, rinkinys domeno objektų, kurie gali būti laikomi kaip visuma.

	\item \textbf{Derinimas} (angl. debugging) - riktų bei klaidų paieška programinėje įrangoje bei jų taisymas.

	\item \textbf{Esybė} (angl. entity) - kažkas, kas egzistuoja pats savaime, faktiškai arba hipotetiškai.

	\item \textbf{Horizontalus išplečiamumas} (angl. horizontal scaling) – galimybė sujungti daugybę techninės ar programinės įrangos esybių taip, jog jos dirbtų kaip visuma. Pavyzdžiui, galima pridėti keletą serverių pasinaudojant grupavimu arba apkrovos paskirstymu taip pagerinant sistemos našumą bei prieinamumą.

	\item \textbf{Metaduomenys} (angl. metadata) - duomenys apie kitus duomenis.

	\item \textbf{Tapatybės funkcija} (angl. identity function) - funkcija be jokio poveikio (ji visada grįžta ir jos argumentai tos pačios reikšmės).

	\item \textbf{Valentingumas} (angl. arity) - funkcijos valentingumas yra argumentų kiekis, kurį ji priima.

\end{itemize}