\section{Nežinau ar naudinga!!!!}

Čia esu parašęs, bet kalba eina apie animaciją ir pranašumus joje. Jeigu verta tai paliksiu kažką.

\subsubsection{Plačiau apie Fran}

Anot \cite{ElliottHudak97:Fran} Interactyvios multimedijos animacijų kūrimas (įskaitant audio, nuotraukas, video, 2D ir 3D grafiką) ilgai buvo sudėtingas ir nuobodus procesas. Tikima, jog sunkumas kyla dėl pakankamai aukšto lygio abstrakcijų nebuvimo, ir ypač dėl sunkumo atskirti modeliavimo ir prezentacijos lygmenis arba kitais žodžiais, tarp to kas yra animacija ir kaip ji turėtų būti atvaizduota. Dėl šios priežasties, programos turi išreikštinai valdyti bendrus realizacijos detales, kurios neturi nieko bendro su animacijos turiniu, o ne patį atvaizdavimą naudojantis žemo lygio vaizduoklio bibliotekas. Šios realizacijos detalės apima:

\begin{itemize}

	\item modeliavimą ir kadrų generavimą pažingsniui keliaujant laiku nepaisant to, jog animaciją yra iš esmės tolydi;

	\item judesio įvesties įvykių sekų surinkimą ir apdorojimą nepaisant to, jog judesio įvestis iš esmės yra tolydi;

	\item laiko dalijimą atnaujinant kiekvieną laike besikeičiančią animacijos parametrą nepaisant to, jog šie parametrai iš esmės lygiagrečiai skiriasi.

\end{itemize} 

Leidžiant programuotojams išreikšti "kas" yra interaktyvi informacija, kažkas gali tikėtis automatizuoti "kaip" tai atvaizduojama. Šiuo požiūriu, neturėtų būti netikėta, jog rinkinys išraiškingų rekursyvių duomenų tipų sujungtų su deklaratyvia programavimo kalba leidžia patogiai modeliuoti animacijas, priešingai nei bendrinė praktika naudoti imperatyvias kalbas sutartinai mišriam modeliavimo/prezentacijos stiliui. Taipogi yra rasta ne griežta semantika, aukštesnės eilės funkcijos, stiprus polimorfinis tipizavimas ir sistemingas perkrovimas yra vertingos kalpos sąvybės, leidžiančios palaikyti sumodeliuotą animaciją. Dėl šių priežasčių, Fran suteikia duomenų tipus programavimo kalboje Haskell.

\subsubsection{Modeliavimo privalumai lyginant su prezentacija}

Modeliavimo privalumai prieš animaciją yra panašūs į funkcinės (arba galima sakyti deklaratyvios) programavimo kalbos paradigmą ir apima aiškumą, kūrimo lengvumą, komponavimą ir švarią semantiką. Be šių yra programai būdingų privalumų, tam tikrais atvejais patrauklesnių iš programinės įrangos kūrėjos bei galutinio vartotojo perspektyvos. Šie privaluai apima:

\begin{itemize}

	\item Kūrimas - turinio kūrimo sistemos natūraliai konstruoja modelius, nes šių sistemų galutinis vartotojas mąsto modelio terminais ir paprasta neturi nei noro nei patirties programavimo prezentacijos detalėse.

	\item Optimizuojamumas - modeliu paremtos sistemos turi prezentacijos subsistemą, kuri gali atvaizduoti bet kokį modelį, kuris gali būti sukurtas sistemoje. Egzistuoja daug galimybių optimizacijai, nes aukšto lygio informacijos detalės yra prieinamos prezentacijos subsistemai.

	\item Reguliavimas - prezentacijos subsistema gali lengviau apibrėžti detalių išsamumo lygio valdymą bei pavyzdžių ėmimo dažnį, būtiną interaktyvioms animacijoms, remiantis reginio sudėtingumo, mašinos greičiu ir apkrova ir t.t.

	\item Mobilumas ir saugumas - modeliavimo platformos nepriklausomumas palengvina mobilių aplikacijų, kurios yra įrodytai saugios WWW(World Wide Web) programos, konstravimą.

\end{itemize}

\subsubsection{Modeliavimo esmė}

Yra keturios pagrindinės modeliavimo idėjos:

\begin{itemize}

	\item Laikinas modeliavimas. Reikšmės, vadinamos elgesiu, kurios kinta bėgant laikui yra labiausiai dominančios. Elgesys yra pirmos klasės reikšmės ir sukurtos kompoziciškai. Lygiagretumas yra išreikštas natūraliai ir neišreikštinai. Pavyzdžiui, sekanti išraiška išreiškia animaciją (paveikslėlio elgesį), kas yra apskritimas ant kvadrato. Laiko taške t, apskritimas turi dydį sin t ir kvadratas turi dydį cos t.

\begin{lstlisting}
	bigger (sin time) circle 'over' bigger (cos time) square
\end{lstlisting}

	\item Įvykių modeliavimas. Kaip ir elgesys, įvykiai yra pirmos eilės reikšmės. Įvykiai gali reikšti tam tikrus nutikimus realiame pasaulyje (pavyzdžiui, pelės mygtuko paspaudimas) arba predikatus paremtus animacijos parametrais (pavyzdžiui, artimumą arba susidūrimą). Tokie įvykiai gali būti sujungti su kitais iki norimo sudėtingumo taip atskiriant sudėtingą animacijos logiką į semantiškai turiningus, modulius konstravimo blokus. Pavyzdžiui, įvykis, aprašantis pirmą kairio mygtuko paspaudimą po laiko t0 yra paprasčiausiai \textit{1bp t0}; aprašantis laiko kvadratą lygų penkiems yra \textit{predicate (pow(time, 2) == 5 t0)} ir jų loginė disjunkcija \textit{1bp t0 .|. predicate (pow(time, 2) == 5) t0}

	\item Deklaratyvus reaktyvumas. Elgesys dažnai yra natūraliai išreiškiamas kaip atsakas į įvykį. Bet netgi šis reaktyvus elgesys turi deklaratyvią semantiką dėl būsenos pasikeitimų, neretai įtraukiamų į įvykiais paremtą formalizmą. Pavyzdžiui, spalvos reikšmės elgesys, kuris periodiškai keičiasi iš raudonos į žalią su kiekvienu mygtuko paspaudimu gali būti aprašytas kaip paprastas pasikartojimas:

\begin{lstlisting}
	colorCycle t0 =
		red 'untilB' 1bp t0 *=> \\t1 ->
		green 'untilB' 1bp t0 *=> \\t1 ->
		colorCycle t2
\end{lstlisting}

	\item Polimorfinė medija. Laike besikeičiančių medijų (nuotraukos, video, garsas, 3D grafika) įvairovė ir šių tipų parametrai (erdvinės transformacijos, spalvos, taškai, vektoriai, skaičiai) turi savo pačių specialiai tipui opecijas (pavyzdžiui, nuotraukų sukimas, garso maišymas, skaitmeninė sudėtis), tačiau sutelpa į bendrinį elgesio ir reaktyvumo karkasą. Pavyzdžiui, 'untilB' operaciją naudojama prieš tai yra polimorfinė, tinkanti bet kuriai laike beisikeičiančiai reikšmei.

\end{itemize}